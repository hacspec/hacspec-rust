\documentclass[10pt, english, a4paper]{article}

\usepackage[utf8]{inputenc}
\usepackage[T1]{fontenc}
\usepackage{lmodern}
\usepackage[english]{babel}
\usepackage{fullpage}
\usepackage{minted}
\usepackage{amsmath, amssymb}
\usepackage[colorlinks]{hyperref}
\usepackage{mathpartir}

\title{The Rustspec language}
\author{Denis Merigoux, Inria}

\newcommand{\rust}[1]{\mintinline[mathescape=true]{rust}{#1}}

\begin{document}
\maketitle

The Rustspec language is a subset of the Rust programming language designed
for writing concise cryptographic specifications.

\section{Core language}

The idea of Rustspec is to isolate a subset of Rust that would behave as an affine,
functional, state-passing programming language. For that, we simply forbid
any kind of borrowing as well as reference types and \rust{mut} variables.
In that setting, we claim that the resulting programming language behaves
like  an extension of the linear line calculus of section 1.2 of \cite{pierce2005advanced}
with the unrestricted terms corresponding to Rust types implementing the \rust{Copy}
trait.

We formalize the Rustspec-core language, a strict subset
of Rust that minimally demonstrate our claim. The syntax of the language is
described in \autoref{fig:rustspec-core-syntax}.

We then show that Rustspec-core programs can be translated to a standard
affine lambda calculus.


\begin{figure}
\begin{align*}
\text{Crate } \rust{$k$}  &::=\quad \varnothing \quad|\quad i\cdot k &\text{List of items}\\
\text{Item } \rust{$i$}     &::=\quad \rust{static $x$ : $t$ = $l$}&\text{Static variables} \\
                            &\quad\;\;|\quad \rust{fn $f$ ( $x$ : $t$ ,..., $x$ : $t$ ) -> $t$ { $e$ }}&\text{Function}\\
\text{Type }\rust{$t$}      &::=\quad \rust{$c$}\\
                            &\quad\;\;|\quad \rust{$d$}\\
\text{Copyable type }\rust{$c$} &::=\quad \rust{bool}&\text{Booleans and integers}\\
                            &\quad\;\;|\quad \rust{[ $c$ ; $n$ ]}&\text{Fixed-size arrays}\\
\text{Non-copyable type }\rust{$d$} &::=\quad \rust{Array< $t$ >}&\text{Variable-size arrays} \\
\text{Expression }\rust{$e$}&::=\quad \rust{$x$}&\text{Variables} \\
                            &\quad\;\;|\quad\rust{$l$}&\text{Literals}\\
                            &\quad\;\;|\quad  \rust{let $x$ = $e$; $e$}&\text{Let bindings} \\
                            &\quad\;\;|\quad \rust{$f$ ( $e$ ,..., $e$ )}&\text{Function calls} \\
                            &\quad\;\;|\quad \rust{if $e$ { $e$ } else { $e$ } }&\text{Conditionals}\\
                            &\quad\;\;|\quad \rust{let mut $x$ = Array::new( $e$ , $e$ ); $e$}&\text{Variable-size arrays}\\
                            &\quad\;\;|\quad \rust{let $x$ = [ $e$ ,..., $e$ ]; $e$}&\text{Fixed-size arrays}\\
                            &\quad\;\;|\quad \rust{let $x$ = $x$ [ $e$ ]; $e$}&\text{Array indexing}\\
                            &\quad\;\;|\quad \rust{$x$ [ $e$ ] = $e$ ; $e$}&\text{Array update}\\
                            &\quad\;\;|\quad \rust{$x$.len()}&\text{Array length}\\
\text{Literal }\rust{$l$}&::=\quad \rust{false}\quad|\quad\rust{true}&\text{Booleans}\\
\end{align*}
\caption{Syntax of Rustspec-core\label{fig:rustspec-core-syntax}}
\end{figure}

\begin{figure}
\begin{align*}
\text{Qualifer }\mathsf{q} &::=\quad\mathsf{lin}&\text{Linear}\\
                           &\quad\;\;|\quad\mathsf{un}&\text{Unrestricted}\\
\text{Boolean }\mathsf{b}  &::=\quad \mathsf{true}\quad|\quad\mathsf{false}&\text{Booleans}\\
\text{Term }\mathsf{t}     &::=\quad x &\text{Variables}\\
                           &\quad\;\;|\quad\mathsf{q}\;\mathsf{b}&\text{Booleans}\\
                           &\quad\;\;|\quad \mathsf{if}\;\mathsf{t}\;\mathsf{then}\;\mathsf{t}\;\mathsf{else}\;\mathsf{t}&\text{Conditionals}\\
                           &\quad\;\;|\quad \mathsf{q}\;\lambda(x:\mathsf{T}).\;\mathsf{t}&\text{Abstraction}\\
                           &\quad\;\;|\quad\mathsf{t}\;\mathsf{t}&\text{Application}\\
                           &\quad\;\;|\quad\mathsf{q}<\mathsf{t},\mathsf{t}>&\text{Pairs}\\
                           &\quad\;\;|\quad\mathsf{split}\;\mathsf{t}\;\mathsf{as}\;x,y\;\mathsf{in}\;\mathsf{t}&\text{Splitting}\\
                           &\quad\;\;|\quad\mathsf{q}\;\mathsf{array}(\mathsf{t},\ldots,\mathsf{t})&\text{Array creation (all terms)}\\
                           &\quad\;\;|\quad\mathsf{q}\;\mathsf{array}(\mathsf{t};\mathsf{t})&\text{Array creation (default value)}\\
                           &\quad\;\;|\quad \mathsf{swap}(\mathsf{t}[\mathsf{t}], \mathsf{t})&\text{Swapping}\\
                           &\quad\;\;|\quad \mathsf{length}(\mathsf{t})&\text{Array length}\\
\text{Pretype }\mathsf{P}  &::= \quad\mathsf{Bool}&\text{Booleans}\\
                           &\quad\;\;|\quad\mathsf{T}\rightarrow\mathsf{T}&\text{Functions}\\
                           &\quad\;\;|\quad\mathsf{T}*\mathsf{T}&\text{Pairs}\\
                           &\quad\;\;|\quad\mathsf{Array} (\mathsf{T})&\text{Arrays}\\
\text{Type }\mathsf{T}      &::=\quad\mathsf{q}\;\mathsf{P}&\text{Qualified pretype}
\end{align*}
\caption{Syntax of the affine lambda calculus with arrays}
\end{figure}

\begin{figure}
\begin{align*}
\text{Unrestricted variable translation context }\Gamma &::= \quad\varnothing\quad|\quad \rust{$x$}\rightsquigarrow x,\;\Gamma
                               \quad|\quad \rust{$f$}\rightsquigarrow x,\;\Gamma\\
\text{Linear variable translation context }\Delta &::= \quad\varnothing\quad|\quad \rust{$x$}\rightsquigarrow x:\mathsf{T},\;\Delta
\end{align*}

\newcommand{\transexpr}[4]{#1\;;\;#2\vdash #3\rightsquigarrow #4}
\newcommand{\transtype}[2]{#1\rightarrowtail #2}

\begin{mathpar}
\inferrule[T-Bool]{\;}{\transtype{\rust{bool}}{\mathsf{un}\;\mathsf{Bool}}}

\inferrule[T-ConstArray]{
  \transtype{\rust{$c$}}{\mathsf{T}}
}{
  \transtype{\rust{[ $c$ ; $n$ ]}}{\mathsf{un}\;\mathsf{array}(\mathsf{T})}
}

\inferrule[T-DynArray]{
  \transtype{\rust{$t$}}{\mathsf{T}}
}{
  \transtype{\rust{Array< $t$ >}}{\mathsf{lin}\;\mathsf{array}(\mathsf{T})}
}

\\

\inferrule[L-True]{\;}{\transexpr{\Gamma}{\Delta}{\rust{true}}{\mathsf{un}\;\mathsf{true}}}

\inferrule[L-False]{\;}{\transexpr{\Gamma}{\Delta}{\rust{false}}{\mathsf{un}\;\mathsf{false}}}

\inferrule[L-ConstArray]{
  \transexpr{\Gamma}{\Delta}{\rust{$l_1$}}{\mathsf{t}_1}\\
  \cdots\\
    \transexpr{\Gamma}{\Delta}{\rust{$l_n$}}{\mathsf{t}_n}
}{
  \transexpr{\Gamma}{\Delta}{\rust{[ $l_1$ ,..., $l_n$ ]}}{\mathsf{un}\;\mathsf{array}(\mathsf{t}_1,\ldots,\mathsf{t}_n)}
}

\\

\inferrule[E-UnVar]{\;}{\transexpr{\rust{$x$}\rightsquigarrow x,\;\Gamma}{\Delta}{\rust{$x$}}{x}}

\inferrule[E-LinVar]{\;}{\transexpr{\Gamma}{\rust{$x$}\rightsquigarrow x:\mathsf{T},\;\Delta}{\rust{$x$}}{x}}


\inferrule[E-LetVar]{
  \transexpr{\Gamma}{\Delta}{\rust{$e_1$}}{\mathsf{t}_1}\\
  \rust{$e_1$ : $t$}\\
  \transtype{\rust{$t$}}{\mathsf{T}}\\
  x\text{ fresh}\\
  \Delta'= \mathsf{refresh}(\Delta)\\
  \transexpr{\rust{$x$}\rightsquigarrow x,\;\Gamma}{\Delta'}{\rust{$e_2$}}{\mathsf{t}_2}\\
}{
  \transexpr{\Gamma}{\Delta}{\rust{let $x$ = $e_1$; $e_2$}}{}{
    (\mathsf{abstract}(\mathsf{pack}(\mathsf{t}_2,\Delta'), (\rust{$x$}\rightsquigarrow x:\mathsf{T},\Delta')))\;\mathsf{t}_1\;\mathsf{pack}(\Delta)
  }
}

\inferrule[E-LetMutVar]{
  \transexpr{\Gamma}{\Delta}{\rust{$e_1$}}{\mathsf{t}_1}\\
  \rust{$e_1$ : $t$}\\
  \transtype{\rust{$t$}}{\mathsf{T}}\\
  x\text{ fresh}\\
  \Delta'= \rust{$x$}\rightsquigarrow x:\mathsf{T},\;\mathsf{refresh}(\Delta)\\
  \transexpr{\Gamma}{\Delta'}{\rust{$e_2$}}{\mathsf{t}_2}\\
}{
  \transexpr{\Gamma}{\Delta}{\rust{let mut $x$ = $e_1$; $e_2$}}{}{
    (\mathsf{abstract}(\mathsf{pack}(\mathsf{t}_2,\Delta'), \Delta'))\;\mathsf{t}_1\;\mathsf{pack}(\Delta)
  }
}

\inferrule[E-DynArray]{
  \transexpr{\Gamma}{\Delta}{\rust{$e_1$}}{\mathsf{t}_1}\\
  \transexpr{\Gamma}{\Delta}{\rust{$e_2$}}{\mathsf{t}_2}\\
}{
  \transexpr{\Gamma}{\Delta}{\rust{Array::new( $e_1$ , $e_2$ )}}{\mathsf{lin}\;\mathsf{array}(\mathsf{t}_1;\mathsf{t}_2)}
}

\inferrule[E-FunctionCall]{
  \transexpr{\rust{$f$}\rightsquigarrow x, \Gamma}{\Delta}{\rust{$e_1$}}{\mathsf{t}_1}\\
  \cdots\\
  \transexpr{\rust{$f$}\rightsquigarrow x, \Gamma}{\Delta}{\rust{$e_n$}}{\mathsf{t}_n}\\
}{
  \transexpr{\rust{$f$}\rightsquigarrow x, \Gamma}{\Delta}{\rust{$f$ ( $e_1$ ,..., $e_n$ )}}{x\;\mathsf{t}_1\;\cdots\;\mathsf{t}_n}
}
%
% \inferrule[E-If]{
%   \transexpr{\Gamma}{\rust{$e_1$}}{\mathsf{t}_1}\\
%   \transexpr{\Gamma}{\rust{$e_2$}}{\mathsf{t}_2}\\
%   \transexpr{\Gamma}{\rust{$e_3$}}{\mathsf{t}_3}\\
% }{
% \transexpr{\Gamma}{\rust{if $e_1$ { $e_2$ } else { $e_3$ }}}{\mathsf{if}\;\mathsf{t}_1\;\mathsf{then}\;\mathsf{t}_2\;\mathsf{else}\;\mathsf{t}_3}
% }

\end{mathpar}
\caption{Translation of Rustspec-core to an affine lambda calculus\label{fig:rustspec-core-translation}}
\end{figure}





\bibliographystyle{plain}
\bibliography{rustspec}

\end{document}
